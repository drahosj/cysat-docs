% platform.tex - Cysat Software Platform API and Payload Interface Document

% Jake Drahos - December 2014


\documentclass{article}
\usepackage{times}
\usepackage[margin=0.75in]{geometry}
\usepackage{hyperref}

\hypersetup{pdfborder={0 0 0}}

\begin{document}

\title{Cysat Platform API Documentation}
\author{Jake Drahos \\ \texttt{drahos@iastate.edu}}
\date{\today\\v0.1 (DRAFT)}

\maketitle

\tableofcontents

\section{Introduction}
This document serves two purposes. The first is to detail the functionality 
provided by the CySat platform and
describe the APIs to access that functionality.

The second purpose of this document is to fully explain the layer of connection
between the payload handling thread and the rest of the CySat platform, namely
the communications/downlink thread. This document will act as a guide for
payload integration, detailing the interfaces that the payload thread should
 use
to extract data from the payload board, and how the payload thread should 
communicate this data to the communications thread.

This document is currently a working draft. Proposals from any team can be taken
into account.

\section{Pin and Peripheral Mapping}

SECTION UNDER CONSTRUCTION - PENDING FINAL STM32 MCU MODULE DESIGN

This section details which ports and pins map to which CubeSatKit IO lines,
as well as which STM32 peripherals are used for the CubeSatKit standard 
interfaces, such as System I2C, System SPI, and UART1/2.

\section{Hardware API}
There is not a full true hardware abstraction layer, but hardware should be
accessed through these API functions if available, rather than through manual
direct register manipulation or through the ST Peripheral Library. This API
contains RTOS wrapping and wrapping to go from CubeSatKit naming conventions
to the actual peripherals.

If a hardware function does not have APIs documented here, it is likely
accessible directly. If in doubt, check the code. GPIOs will always be directly 
accessible, but check
to see if the pins are in use.

\subsection{Serial Peripheral Interface (SPI) API}
The SPI peripherals are wrapped in a DMA and RTOS layer to allow the current
task to block while transferring, yet other tasks to run in the meantime.
Mutexes are used to avoid peripheral conflicts. It is conceivable to take the
mutex for a SPI peripheral and then directly access it, but that is not
recommended.

The functions documented here refer to SPIx, where x is a number. These numbers
correlate to the STM32 peripheral names, not the CubeSatKit bus names. See
the above section ``Pin and Peripheral Mapping'' to associate STM32 peripherals
with CubeSatKit bus interfaces.

Core API Functions
\begin{center}
\begin{tabular}{| l | l |}
    \hline 
    Function & Description \\ \hline
    SPIxMutexTake(timeout) & Take mutex for SPIx.  \\ \hline
    SPIxMutexRelease() & Release mutex for SPIx. \\ \hline
    SPIxBlockingTransfer(*tx, *rx, length) & Perform a blocking DMA'd transfer \\ \hline
\end{tabular}
\end{center}

SPIxMutexTake(timeout) returns true if the mutex was acquired or false if the
timeout (in msec) was reached. Once the mutex is acquired, it must be released
with SPIxMutexRelease()
once the task is no longer using it. This mutex is held separately of the
transfer to allow operations such as Flash status checking to be done within the
same ``session'' as an actual read or write. This mutex should not be used for
large sequences of complex reads or writes, such as filesystem access. A separate
mutex should be used for functionality built on top of this layer.
    
While the mutex is held, SPIxBlockingTransfer can be used to perform a SPI session.
All that this function does is kick off a DMA session, sequentially transmitting
bytes from the tx buffer and storing the received bytes in the rx buffer until
length bytes have been transmitted. Any assertion of CS lines must be manually
done. The current task will block while the transfer is in progress, but other
tasks will be allowed to run. Both pointer arguments must point to valid buffer
space of at least length.

The following snipped shows an example of writing the sequence 0x3, 0x2, 0x1 to
a hypothetical external device, and then reading the middle byte back.
The hypothetical device takes one byte of opcode, three bytes of address, and then
an arbitrary number of data bytes, either read or written.
The stub functions
assertChipSelect() and deassertChipSelect() would be implemented by the task in question,
and the opcodes READ\_OPCODE and WRITE\_OPCODE are assumed to be defined.

\begin{verbatim}
/* Build txBuffer to write the string 0x3, 0x2, 0x1 at address 0x0000 */
uint8_t txBuffer[6] = {WRITE_OPCODE, 0x00, 0x00, 0x3, 0x2, 0x1}
uint8_t rxBuffer[6]; /* Empty read buffer */

if (SPI1MutexTake(10000)) /* Delay up to 10s taking mutex */
{  
    assertChipSelect(); /* Drive CS low to initiate SPI transaction
    /* Transmit WRITE_OPCODE, ADDHIGH, ADDRLOW, DATA, DATA, DATA */
    SPI1BlockingTransfer(txBuffer, rxBuffer, 6);
    deassertChipSelect();
    
    /* The hypothetical device has now written 0x3, 0x2, 0x1 at address 0x0000.
    txBuffer is unaltered, and rxBuffer is full of junk */
    
    
    txBuffer[0] = READ_OPCODE;
    txBuffer[2] = 0x01;
    
    /* The transmit buffer is now pre-filled to contain a 
    READ ADDRESS 0x0001 instruction, followed by sufficient dummy data bytes. 
    Note that the data bytes could be any value, as the hypothetical device 
    operates in half-duplex mode */
    
    assertChipSelect();
    /* Transmit READ_OPCODE, ADDRESS, DUMMY */
    SPI1BlockingTransfer(txBuffer, rxBuffer, 4);
    deassertChipSelect();
    
    /* rxBuffer now contains [junk, junk, junk, 0x01] */
    
    SPI1MutexRelease();
}
else
{
    /* Handle error */
}

\end{verbatim}

In the future, non-blocking transfer functions may be added if necessary.

\subsection{Inter-Inegrated Circuit (I2C)}
The I2C API is very similar to the SPI API, however it features only one buffer
and separate functions for reads and writes.

Core API Functions
\begin{center}
\begin{tabular}{| l | l |}
    \hline 
    Function & Description \\ \hline
    I2CxMutexTake(timeout) & Take mutex for I2Cx.  \\ \hline
    I2CxMutexRelease() & Release mutex for I2Cx. \\ \hline
    I2CxBlockingRead(address, *buffer, length) & Perform a blocking DMA'd read \\ \hline
    I2CxBlockingWrite(address, *buffer, length) & Perform a blocking DMA'd write \\ \hline
\end{tabular}
\end{center}

The I2C API is used much like the SPI api, however there is only one buffer that
is used to store bytes to be written or bytes to be read in the case of a write
or a read, respectively. Once the mutex is obtained, to perform a read simply
call I2CxBlockingRead() with the appropriate address, buffer, and length. To 
perform a write, pre-fill the buffer with the data to be written, including any
opcodes, but NOT including the I2C device address. Call I2CxBlockingWrite() with
the appropriate arguments to perform the transfer. Release the mutex when complete.