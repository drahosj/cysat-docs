% platform.tex - Cysat Software Platform API and Payload Interface Document

% Jake Drahos - December 2014


\documentclass{article}
\usepackage{DejaVuSerif}
\usepackage[margin=0.75in]{geometry}
\usepackage{hyperref}
\usepackage{caption}

\hypersetup{pdfborder={0 0 0}}

\begin{document}

\title{Cysat Platform API and Payload Interface Documentation}
\author{Jake Drahos \\ \texttt{drahos@iastate.edu}}
\date{\today\\v0.3 (DRAFT)}

\maketitle

\tableofcontents

\section{Introduction}
This document serves two purposes. The first is to detail the functionality 
provided by the CySat platform and
describe the APIs to access that functionality.

The second purpose of this document is to fully explain the layer of connection
between the payload handling thread and the rest of the CySat platform, namely
the communications/downlink thread. This document will act as a guide for
payload integration, detailing the interfaces that the payload thread must
 use
to extract data from the payload board, and how the payload thread will 
communicate this data to the communications thread.

This document is currently a working draft. Proposals will be taken into account
and adjustments made. Also note that, until this document leaves draft status,
the API documentation is not intended to be an exhaustive or canonical
reference. Rather, it is intended to give an idea of how the APIs work. As
development proceeds, new APIs can and will be added to expose additional
functionality, but these APIs will operate in much the same way as those
documented here.

\section{Pin and Peripheral Mapping}

SECTION UNDER CONSTRUCTION - PENDING FINAL STM32 MCU MODULE DESIGN

This section details which ports and pins map to which CubeSatKit IO lines,
as well as which STM32 peripherals are used for the CubeSatKit standard 
interfaces, such as System I2C, System SPI, and UART1/2.

\section{Hardware API}
There is not a full true hardware abstraction layer, but hardware must be
accessed through these API functions if available, rather than through manual
direct register manipulation or through the ST Peripheral Library. This API
contains RTOS wrapping and wrapping to go from CubeSatKit naming conventions
to the actual peripherals. Some hardware functionality is directly accessible.
\\

The Doxygen-generated reference documentation is a companion to this section.

\subsection{General Purpose IO (GPIO)}
Since GPIO access is rather fundamental, no wrapping or driver will exist. Conflict
resolution will be handled via a portmap/pin assignment spreadsheet.
\subsubsection{Driver API}
None. Use ST Peripheral Library or direct register manipulation.

\subsection{Serial Peripheral Interface (SPI)}
The SPI peripherals are wrapped in a DMA and RTOS layer to allow the current
task to block while transferring, yet other tasks to run in the meantime.
For this reason, it is very important to use the provided driver to avoid
unnecessary blocking or major loss of throughput due to context switching.
\subsubsection{Driver API}
See the CySat\_SPI module in the Doxygen-generated documentation for detailed
information.
\subsubsection{Numbering Scheme}
The SPI API functions refer to SPIx, where x is a number. These numbers
correlate to the STM32 peripheral names, not the CubeSatKit bus names.

\subsection{Inter-Inegrated Circuit (I2C)}
The I2C API is very similar to the SPI API.
\subsubsection{Driver API}
See the CySat\_I2C module in the Doxygen-generated documentation for detailed
information.
\subsubsection{Numbering Scheme}
The numbered I2Cx peripherals use the STM32 name, not CubeSatKit names.

\subsection{USART/Serial}
UART access is handled through the UART Rx and Tx threads. Queues act as
buffers in a manner similar to many desktop operating systems, except that
there are no per-thread buffers.
\subsubsection{Driver API}
Documented in the CySat\_UART module of the Doxygen documentation
\subsubsection{Numbering Scheme}
The UARTx in the UART API follow CubeSatKit conventions. UART1 is the console
UART and should only be using to print debugging information. UART2 is the radio, and
should not be directly accessed. There is also a set of UART functions prefixed ``vConsole''.
These functions should be preferred for console debug and informational messages.

\subsection{Real-Time Clock}
The real-time
clock is wrapped to provide a mission clock.
\subsubsection{Driver API}
Documented in the CySat\_Clock module of the Doxygen documentation.

\subsection{Backup Registers}
The STM32 RTC peripheral contains a real-time clock as well as 
backup registers that exist in the backup power domain. The backup registers are 
used as nonvolatile state flags. Since the availability of backup registers
is extremely limited, there is no public API. Additionally, only the
initialization thread will have a persistent state, as well as some 
specific configuration data that must persist, so for design reasons
access to the backup registers is limited. Should a need for the backup
registers arise, they can be allocated to the payload thread, and direct
access will be used.
\subsubsection{Driver API}
None. Use should be limited for design reasons as well as limited resources. Direct
access will be used in situations where persistence is deemed necessary and
registers can be spared.

\subsection{Reset and Clock Control}
The RCC peripheral of the STM32 is largely responsible for power management
and enabling/disabling of specific peripherals. The RCC 
should not be used except
by drivers, or when the payload thread 
is granted direct access, without contention,
to an entire peripheral. Any peripheral noted as ``not utilized'' can
be monopolized by the payload thread. 
In this case, the payload thread is free to
power the monopolized peripheral on or off as needed.
\subsubsection{Enabling Peripherals}
Any peripheral that is directly accessed must have its corresponding 
RCC Enable bit set as part of initialization. However, do not disable
the RCC Enable bit as part of deinitialization. This is especially
true of GPIOs. If a GPIO port will be used, enable it in the
RCC as part of initialization, but never disable a GPIO port in
the RCC.

\subsection{DMA Controller}
The DMA controller is heavily leveraged by the I2C and SPI drivers, but it may
come into use for other threads. In that case, a DMA stream may be allocated
to that thread.
\subsubsection{Driver API}
None. Direct access is used.
\subsubsection{DMA Allocation}
\begin{tabular}{| l | c |}
    \hline 
    DMA Streams Allocated to Drivers & DMA2\_Stream2, DMA2\_Stream5, DMA1\_Stream5, DMA1\_Stream6\\ \hline
\end{tabular}

\subsection{ADC}
The ADC is not currently utilized by the CySat platform. It can currently be directly
accessed by any thread.
\subsubsection{Driver API}
None. The ADC is not currently used.

\subsection{DAC}
The DAC is not currently utilized by the CySat platform. It can currently
be directly accessed by any thread.
\subsubsection{Driver API}
None. The DAC is not currently used.

\subsection{Timers}
The hardware timers should not be used. RTOS Task Delays and timers should be used instead.

\subsection{Hash Processor}
The cryptographic processor is used to validate signatures for sensitive commands. Since it
is a rather unique use-case peripheral, direct access will be used, but it will be
protected by a RTOS mutex.
\subsubsection{Driver API}
Mutex protection documented in Doxygen. Direct access for use.

\subsection{Cryptographic Processor}
The cryptographic processor is not currently utilized.
\subsubsection{Driver API}
None. The cryptographic processor is not utilized.

\subsection{Random Number Generator}
The RNG is not currently utilized.
\subsubsection{Driver API}
None. The RNG is not currently utilized.

\subsection{SDIO Interface}
The SDIO interface is an alternative to the SPI interface for SD card access.
Should SDIO be used, it will be implemented as a subset of the SPI driver.
\subsubsection{Driver API}
Not currently implemented, but would be implemented as a sub-feature of the 
SPI driver.

\subsection{Flexible Memory Controller (FMC) and Flexible Static Memory Controller}
The FSMC is used as part of the housekeeping logging thread. It must 
not be used by 
the payload thread. If the payload thread has housekeeping data that needs to
be logged,
an interface to the housekeeping thread can be developed.
\subsubsection{Driver API}
Should not be accessed by payload thread. If the payload thread has housekeeping data,
an interface to the housekeeping thread can be created.

\subsection{USB on-the-go}
Not implemented and should not be used.

\subsection{CAN}
Not implemented and should not be used.

\subsection{Ethernet}
Not implemented and should not be used.

\section{Platform-Payload Interface}
\subsection{Proposal}
This section is a proposal by the Iowa State team, not
a final set of requirements. It is based on the current understanding of the
payload's requirements and capabilities, as well as the current state and 
capabilities of the CySat platform. This section can be expected
to undergo heavy revision until this is no longer a draft or
a proposal.


\subsection{Payload Thread}
All payload interface will be handled by a single subsystem, referred to as the
``payload thread.'' As such, the payload-platform interface can be abstracted
to an interface between the payload thread and several other threads, namely
the communications thread and initialization thread.

\subsection{Payload Thread Scope and Responsibilities}
The payload thread will have the following responsibilities:
\begin{itemize}
    \item Filesystem.
    \item SD Card access arbitration.
    \item Image selection.
    \item Payload board initialization and management.
    \item Presentation of images to the communications thread.
\end{itemize}

\subsubsection{Filesystem}
The filesystem will be entirely the domain of the payload thread. It will not be
used for any other purpose. As such, the filesystem implementation is completely
internal to the payload thread. However, the payload thread must use the
platform APIs (SPI or SDIO) for SD card access to ensure that other threads will not be blocked
during large reads or writes and to avoid conflicts with other external hardware
that may be on the same bus.

\subsubsection{SD Card Access Arbitration}
SD card access arbitration will be done by the payload thread using direct access
to the GPIOs. The system I2C driver can also be used for arbitration.

\subsubsection{Image Selection}
Image selection will be handled by the payload thread. This process must be
non-volatile, ie. if there is a power loss, the same images must be re-loaded
upon restart. Ideally this will be handled by the flash filesystem, however
access can be granted to the backup registers if this would simplify the
process.

\subsubsection{Payload board initialization, configuration, and management}
The payload thread will initialize the payload board as part of its initialization.
Initialization will be done upon startup, however the payload thread and payload
board must be initialized into a low-power, inactive state. The payload thread
and payload board must be able to be activated and deactivated many times
throughout the mission timeline.

\subsubsection{Image loading and presentation}
The largest responsibility of the payload thread will be to load images from 
the SD card and present them to the communications thread for downlinking. This
is described in detail below.

\subsection{Communication Between Threads}
The full interface is defined in payload.h. See the header file for details.
\subsubsection{Signal Queues}
Two queues exist and pass signals between the payload thread and the
communications thread. The payload thread calls ReceiveCommunicationsEvent()
periodically to check for CommunicationsEvents from the
communications thread. The payload thread calls PostPayloadEvent() to post 
PayloadEvents to the Communications thread. The events are of the type
PayloadEvent\_t and CommunicationsEvent\_t. Each event has a signal and a 32-bit
integer attribute. The signal types are defined in payload.h, as well as the
contextual meaning of the attribute integer.

\subsubsection{Image Slots}
A number of image ``slots'', 64KB in size, are statically allocated,
along with a number of flags  used to communicate state from the payload thread
to the communications thread. These slots will be
populated by the payload thread and read by the communications thread. Upon
command from the communications thread, the payload thread must ``flush'' the
specified slot, deleting that image from the SD card and then proceeding to
load a new image into its place. If an image is over 64KB in size, it must be
discarded, preferably by the payload board. The payload thread must not present
an image over 64KB in size to the communications thread.

The payload thread must post the appropriate signals upon any image actions
as well as keep the value of GetImagesOnSDCard() up to date as images are
loaded and flushed.

\subsubsection{Payload Configuration Settings}
{\bfseries Note: The examples in this section are a placeholder to demonstrate
how the settings system works. The exact settings available and meaning of
different values must be finalized by project leadership}

Configuration change requests are sent as CommunicationsEvents with the 
appropriate signal type and relevant attribute set to an enum value. The 
current value of a configuration setting is read by the payload thread by
calling functions such as GetPayloadResolution() or GetPayloadCaptureFrequency().
These functions read the global configuration state stored by the payload
and return it.

\subsubsection{Global Variables and Functions}
Image slots are globally accessible as an array of ImageSlot\_t structs. The
event queues should not be directly accessed; call the Receive and Post
functions in the header file. RTOS Queues are pass-by-copy.

{\bfseries See payload.h for more details.}

\subsection{State Definitions}
There are three states: uninitialized, inactive, and active.

\subsubsection{Uninitialized}
The payload board begins in the uninitialized state until initialization.
Variables are undefined, and the payload board is in an undefined/unknown state.
The uninitialized state can be the result of a complete power loss or a graceful
reboot, so no assumptions as to the state of the payload board can be made.

\subsubsection{Inactive State}
The payload board is in the low-power mode during the inactive state. The
payload thread still responds to any configuration changes and polling as
appropriate. Any configuration change commands are applied to the global
configuration state, and any reads of the global configuration state take any
such
commands into account. However, the payload board remains in the low-power mode
and no SD card operations are performed until the active state is entered. The
inactive state is considered the safe state for satellite reboots with regards
to the payload board and SD card.

\subsubsection{Active State}
In the active state, the payload board is taking pictures and the payload thread
is loading images into the SD card. The payload board is loaded with the current
global configuration state upon activation, and any configuration change
commands received when in the active state are immediately applied to both
the global state and the payload board.
In order to obtain a 
``pseudostate'' where the payload thread loads images, but the payload board
does not capture new images, enter the active state with the image capture
frequency set to a ``none'' or ``disabled'' seting.

\subsection{State Definitions}
There are three states: uninitialized, inactive, and active.

\subsubsection{Uninitialized}
The payload board begins in the uninitialized state until initialization.
Variables are undefined, and the payload board is in an undefined/unknown state.The uninitialized state can be the result of a complete power loss or a graceful
reboot, so no assumptions as to the state of the payload board can be made.

\subsubsection{Inactive State}
The payload board is in the low-power mode during the inactive state. The
payload thread still responds to any configuration changes and polling as
appropriate. Any configuration change commands are applied to the global
configuration state, and any reads of the global configuration state take any
such
commands into account. However, the payload board remains in the low-power mode
and no SD card operations are performed until the active state is entered.

\subsubsection{Active State}
In the active state, the payload board is taking pictures and the payload thread
is loading images into the SD card. The payload board is loaded with the current
global configuration state upon activation, and any configuration change
commands received when in the active state are immediately applied to both
the global state and the payload board.
In order to obtain a 
``pseudostate'' where the payload thread loads images, but the payload board
does not capture new images, enter the active state with the image capture
frequency set to a ``none'' or ``disabled'' seting.

\subsection{Normal Operation}
\subsubsection{Initialization}
Upon initialization, the payload thread must set all flags and configuration
settings to sane defaults. No persistent state should be taken into account. Upon
a power loss and reboot, the payload thread and board must be in an inactive state
with sane default settings. As part of initialization, the payload thread must
ensure that the payload board is in the low-power state.

\subsubsection{Activation}
Upon receiving a command to activate, the payload board must
be brought out of low-power mode. As part of being brought out of low-power
mode, the payload board should be loaded with the current global configuration
state. The payload thread should now begin loading 
images into the slots as soon as they become available. Images should have some
sort of intrinsic priority (such as timestamp) and the highest priority image
should be loaded as soon as a slot is available. This priority system is defined
by the payload thread and the payload board, and has no influence from the
rest of the satellite systems. Once an image is loaded into
imageSlotA, imageSizeA must be set. Once another image becomes available, 
the payload thread should load it into imageSlotB, then set the size for
imageSlotB. Images remain in the slot until removed by a flushImage command.

\subsubsection{Deactivation}
The payload board is put into low-power mode, pending SD card operation. 
Image loading must immediately cease, but SD card operations are allowed to
continue until the filesystem is in a stable, non-corrupt state. The
payload thread must not leave the filesystem in a corrupt state.

\subsubsection{Configuration Changes and the Global Configuration State}
The global configuration state is maintained in main memory at all times. 
This state is initialized to sane
defaults at initialization, modified as necessary at any time, and copied
to the payload board upon activation. 
The global configuration state is not in any specific format, rather it is
a collection of variables accessed by a number of ``get-'' functions. It does
not need to be copied in binary format to the payload board, but should
accurately reflect the functionality of the payload board at all times, ie. for
any given set of values, the payload board should behave in a consistent manner.
When the payload thread receives a configuration change command, such as resolution
or capturefrequency, it must modify the global configuration state as
appropriate, and, if the payload is active, perform the necessary operations
to update the payload board configuration.

\subsubsection{Flushing and Replacing Images}
Upon reception of a flush command, the payload thread must first set the
appropriate image size variable to zero, then delete the image
from the SD card. At this point the slot is
considered empty. The highest priority image (that is not already loaded into
the other slot) should be loaded into this slot
once available, replacing the current buffer contents. Once a new image is fully
<<<<<<< HEAD
loaded into the slot, the appropriate imageSize variable must be set.
=======
loaded into the slot, the appropriate imageSize flag must be set.
>>>>>>> origin/master

\end{document}
