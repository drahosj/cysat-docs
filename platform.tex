% platform.tex - Cysat Software Platform API and Payload Interface Document

% Jake Drahos - December 2014


\documentclass{article}
\usepackage{times}
\usepackage[margin=0.75in]{geometry}
\usepackage{hyperref}
\usepackage{caption}

\hypersetup{pdfborder={0 0 0}}

\begin{document}

\title{Cysat Platform API Documentation}
\author{Jake Drahos \\ \texttt{drahos@iastate.edu}}
\date{\today\\v0.1 (DRAFT)}

\maketitle

\tableofcontents

\section{Introduction}
This document serves two purposes. The first is to detail the functionality 
provided by the CySat platform and
describe the APIs to access that functionality.

The second purpose of this document is to fully explain the layer of connection
between the payload handling thread and the rest of the CySat platform, namely
the communications/downlink thread. This document will act as a guide for
payload integration, detailing the interfaces that the payload thread should
 use
to extract data from the payload board, and how the payload thread should 
communicate this data to the communications thread.

This document is currently a working draft. Proposals will be taken into account
and adjustments made. Also note that, until this document leaves draft status,
the API documentation is not intended to be an exhaustive or canonical
reference. Rather, it is intended to give an idea of how the APIs work. As
development proceeds, new APIs can and will be added to expose additional
functionality, but these APIs will operate in much the same way as those
documented here.

\section{Pin and Peripheral Mapping}

SECTION UNDER CONSTRUCTION - PENDING FINAL STM32 MCU MODULE DESIGN

This section details which ports and pins map to which CubeSatKit IO lines,
as well as which STM32 peripherals are used for the CubeSatKit standard 
interfaces, such as System I2C, System SPI, and UART1/2.

\section{Hardware API}
There is not a full true hardware abstraction layer, but hardware should be
accessed through these API functions if available, rather than through manual
direct register manipulation or through the ST Peripheral Library. This API
contains RTOS wrapping and wrapping to go from CubeSatKit naming conventions
to the actual peripherals.

If a hardware function does not have APIs documented here, it is likely
accessible directly. If in doubt, check the code. GPIOs will always be directly 
accessible, but check
to see if the pins are in use.

\subsection{Serial Peripheral Interface (SPI)}
The SPI peripherals are wrapped in a DMA and RTOS layer to allow the current
task to block while transferring, yet other tasks to run in the meantime.
Mutexes are used to avoid peripheral conflicts. It is conceivable to take the
mutex for a SPI peripheral and then directly access it, but that is not
recommended.

The SPI API functions refer to SPIx, where x is a number. These numbers
correlate to the STM32 peripheral names, not the CubeSatKit bus names. See
the above section ``Pin and Peripheral Mapping'' to associate STM32 peripherals
with CubeSatKit bus interfaces.

See the CySat\_SPI module in the Doxygen-generated documentation for detailed
information.


\subsection{Inter-Inegrated Circuit (I2C)}
The I2C API is very similar to the SPI API, however it features only one buffer
and separate functions for reads and writes.

The numbered I2Cx peripherals are based on the STM32 name, not CubeSatKit names.

See the CySat\_I2C module in the Doxygen-generated documentation for detailed
information.


\subsection{UART/Serial}
The UARTx in the UART API follow CubeSatKit conventions. UART1 is the console
UART, and UART2 is the Radio UART. The functions are all rather self-explanatory.
vConsole(Put/Print/Printf) should be used for console printfs, reserving explicit UART1
and UART2 access for situations where the distinction is significant.

See the CySat\_UART module of the Doxygen-generated documentation for detailed
information.


\subsection{Mission Clock}
getMissionTime(): returns a time\_t of the mission time (seconds since first
power-on).

\subsection{Nonvolatile Storage (State Flags)}
Nonvolatile Storage is available through the RTC's backup registers.
Ideally, the payload thread should be stateless or read any state information
from the SD card. Additionally, the backup registers are extremely limited, as
only 20 4-byte registers are available. For these reasons, no API is provided
at this time. However, should this functionality become necessary, functions
will be added. These functions will be named [name]BackupRegisterRead()
and [name]BackupRegisterWrite(), where [name] is 
based on the subsystem that will use the backup register.
Backup registers will be added by name, rather than address or offset.
This will discourage usage creep, which is a major issue due to the
extremely limited availability of the backup registers.

\section{Platform-Payload Interface}
This section is a proposal by the Iowa State team, not
a final set of requirements.

All payload interface will be handled by a single subsystem, referred to as the
"payload thread." As such, this section largely describes the interface between
the payload thread and the rest of the platform threads.

This section will undergo heavy revision during the draft/feedback process.

\subsection{Payload Thread Scope}
The payload thread will have the following responsibilities:
\begin{itemize}
    \item Filesystem.
    \item SD Card access arbitration.
    \item Image selection.
    \item Payload board initialization and management.
    \item Presentation of images to the communications thread.
\end{itemize}

The filesystem will be entirely the domain of the payload thread. It will not be
used for any other purpose. As such, the filesystem implementation is completely
internal to the payload thread. However, the payload thread must use the
platform APIs (SPI or SDIO) for SD card access to ensure that other threads will not be blocked
during large reads or writes and to avoid conflicts with other external hardware
that may be on the same bus.

SD card access arbitration will be done by the payload thread using direct access
to the GPIOs. The system I2C bus can also be used for arbitration.

Image selection will be handled by the payload thread. This process must be
non-volatile, ie. if there is a power loss, the same images must be re-loaded
upon restart. Ideally this will be handled by the flash filesystem, however
access can be granted to the backup registers if this would simplify the
process.

The payload thread will initialize the payload board as part of its initialization.
Initialization will be done upon startup, however the payload thread and payload
board should be initialized into a low-power, inactive state. The payload thread
and payload board should be able to be activated and deactivated many times
throughout the mission timeline.

The largest responsibility of the payload thread will be to load images from 
the SD card ard present them to the communications thread for downlinking. This
is described in detail below.

\subsection{Image Presentation}
Two image "slots", tentatively 64KB in size, would be statically allocated,
along with a number of flags used to communicate state from the payload thread
to the communications thread. These slots, imageSlotA and imageSlotB, would be
populated by the payload thread and read by the communications thread. Upon
command from the communications thread, the payload thread would "flush" the
specified slot, deleting that image from the SD card and then proceeding to
load a new image into its place.

As stated above, a number of flags will communicate state from the payload
thread to the communications thread. The payload thread will receive from a 
queue to receive commands from the communications thread. These commands would
include the "flush" command above, as well as activate/deactivate, purge, and
miscellaneous configuration information that should be forwarded to the payload
board or otherwise handled appropriately.

\begin{table}
\begin{center}
\caption{Globally accessible Payload-Platform Interface variables}
\begin{tabular}{| l | l |}
    \hline 
    Variable & Brief description \\ \hline
    uint8\_t[] imageSlotA & First 64K image buffer \\ \hline
    uint8\_t[] imageSlotB & Second 64K image buffer \\ \hline
    uint32\_t imageSizeA & Image A size in bytes. 0 for not loaded\\ \hline
    uint32\_t imageSizeB & Image B size in bytes. 0 for not loaded \\ \hline
    uint32\_t imagesOnSDCard & Number of images on SD card. Periodically updated \\ \hline
    payloadStatus\_t payloadStatus & Enum for current payload status (active/inactive) \\ \hline
    resolution\_t currentResolution & Enum for resolution. \\ \hline
    captureFrequency\_t currentCaptureFrequency & Enum for capture frequency. \\ \hline
    xQueueHandle payloadCommandQueue & Queue of commands sent to payload thread \\ \hline
    uint32\_t errorFlags & Bit mask of error flags \\ \hline
\end{tabular}
\end{center}
\end{table}

All of the above variables, with the exception of the queue,
will be written only by the payload thread. To enforce
this, a number of getters could be used and the variables could be made non-global.
If this is done, the getter naming scheme would be getVariableName, where
VariableName is the name given above with the first letter capitalized.
Getters for the image slots are obviously impractical.

\begin{table}
\begin{center}
\caption{Non-exhaustive list of commands}
\begin{tabular}{| l | l |}
    \hline 
    Command & Brief description \\ \hline
    activate & Turn on payload board and begin populating image slots if possible \\ \hline
    deactivate & Turn off payload board and immediately cease population of image slots \\ \hline
    purge & Delete all images from SD card. Ignored while active \\ \hline
    flushSlotA & Flush slot A and delete from SD card \\ \hline
    flushSlotB & Flush slot B and delete from SD card \\ \hline
    setResolutionLow & Set resolution\\ \hline
    setResolutionMed & Set resolution\\ \hline
    setResolutionHigh & Set resolution\\ \hline
    setCaptureFrequencyHighest & Set frequency at which images are taken\\ \hline
    setCaptureFrequencyHigh & Set frequency at which images are taken\\ \hline
    setCaptureFrequencyMedium & Set frequency at which images are taken\\ \hline
    setCaptureFrequencyLow & Set frequency at which images are taken\\ \hline
    setCaptureFrequencyLowest & Set frequency at which images are taken\\ \hline
    setCaptureFrequencyNone & Stop capturing images. Put payload board into low-power mode, but do not deactivate.\\ \hline
\end{tabular}
\end{center}
\end{table}

\subsection{Flow}
Upon initialization, the payload thread should set all flags and configuration
settings to sane defaults.

Upon initialization or deactivation, the payload thread should ensure that the 
payload board is in a low-power mode. While inactive, the payload thread should
still properly handle any configuration changing commands.

Upon receiving a command to activate, the payload board should be powered on (or
brought out of low-power mode) and the payload thread should begin loading 
images into the slots as soon as they become available. Images should have some
sort of intrinsic priority (such as timestamp) and the highest priority image
should be loaded as soon as a slot is available. Once an image is loaded into
imageSlotA, imageSizeA should be set. Once another image becomes available, 
the payload thread should load it into imageSlotB, then set the size for
imageSlotB.

When the payload thread receives a configuration change command, such as resolution
or capturefrequency, it should set the appropriate global flags, and forward
the command to the payload board if appropriate. Depending on the implementation
of "low-power mode" for the payload board, this might not be possible while
inactive. Should that be the case, any configuration information must be copied to
the payload board upon activation.

Upon reception of a flush command, the payload thread should delete the image
from the SD card, then set the image size to 0. At this point the slot is
considered empty. The highest priority image (that is not already loaded into
the other slot) should be loaded into this slot
once available.

\end{document}
